\section{Radial orbital density plots}
The radial orbital density, $\bar{\psi}^2(r)$, plots are created by integrating over angular portion of the norm of the single particle wavefunction.

\begin{equation}
    \bar{\psi}^2(r) = \int_0^{2\pi}\int_0^{\pi} \psi^2(r,\theta,\phi) r^2 \sin(\theta) dr d\theta d\phi
\end{equation}

Discretizing this expression using a uniform radial grid and a Lebedev-Laikov quadrature for the angular components, yields a form that can be readily evaluated.

\begin{equation}\label{eq:plotted}
    \bar{\psi}^2(r_i) = 4 \pi {r_i^2}  \sum_j^{N^{ang}} w^{ang}_j \psi^2(r_i,\theta_j,\phi_j) 
\end{equation}

The function $\bar{\psi}^2(r_i)$ from Equation~\ref{eq:plotted} can be plotted with the points $r_i$ serving as the abscissa.
Since the singly occupied orbitals are normalized, the proximity of the sum of the radial quadrature to unity is used as a check.

\begin{equation}\label{eq:norm}
    \sum_i^{N^{rad}} \bar{\psi}^2(r_i) w^{rad}_i = \sum_i^{N^{rad}} \bar{\psi}^2(r_i) {\Delta}r \approx  1
\end{equation}

\subsection{Dependency versions}
\begin{table}[H]
\begin{tabular}{ll}
dependency      & version     \\
\texttt{numpy}  & 1.18.4      \\
\texttt{quadpy} & 0.16.2      \\
\texttt{pyscf}  & 1.7.0       \\
\texttt{cclib}  & 1.6.3       \\
\end{tabular}   
\end{table}


\subsection{Step 1: Generating a Molden file}
Molden files were generated using \texttt{cclib}, with the exception of the natural orbital from the CIPSI calculations. 
Since QuantumPackage is not supported by \texttt{cclib}, Molden files were created using the native utility in QuantumPackage 2.0. 
For the Molden files generated with cclib, the \texttt{-g/--ghost} flag indicates the presence of a ghost atom.
The \texttt{-n/--naturalorbtials} was created to allow natural orbitals to be written instead of molecular orbitals.
This flag is not yet available in the official distribution, but a request to incorporate it in the official distribution has been opened (\url{https://github.com/cclib/cclib/pull/948}).

%\lstinputlisting[label=cclibmoldenbasic,caption=Using \texttt{cclib} to write a Molden file from an output file. ,language={bash}]{parts/cclibmolden.sh}
\inputminted{zsh}{parts/cclibmolden.sh}

\subsection{Step 2: Integrating over the angular components of the singly occupied orbital}

\texttt{quadpy} was used to generate the Lebedev-Laikov integration weights and points.
The singly occupied molecular/natural orbital was evaluated at these points using \texttt{PySCF}.

%\lstinputlisting[label=anionradialint,caption=Using \texttt{Quadpy} and \texttt{PySCF} to integrate the singly occupied orbital. ,language={Python}]{parts/anionradialint.py}
\inputminted{python}{parts/anionradialint.py}

